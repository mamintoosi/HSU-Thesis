% !TeX root=main.tex
% !TEX TS-program = XeLaTeX
\chapter*{پیش‌گفتار}
\addcontentsline{toc}{chapter}{پیش‌گفتار}

رعایت قانون‌های تدوین شده از جانب نهادهای مسؤول در دانشگاه همچون معاونت آموزشی و معاونت پژوهشی امری الزامی در حروف‌چینی مستندات علمی دانشگاهیان است. یکی از موارد پر استفاده قالب مستندات علمی، نگارش پایان‌نامه است که شامل پروژه‌های دوره کارشناسی، پایان‌نامه‌های دوره ارشد و رساله‌های دکترا می‌شود.
چارچوب کلی  نگارش \پ های دانشگاه حکیم سبزواری توسط نهادهای ذیربط مدون شده و دانشجویان این دانشگاه باید مستندات خود را بر اساس آن آماده نمایند.

پیروی از این قوانین در نرم‌افزاری مانند میکروسافت ورد
\lr{(Microsoft Word)}
  امری زمان‌بر بوده و وقت زیادی هم از دانشجو، هم استاد راهنما و هم مدیریت تحصیلات تکمیلی و کتابخانه دانشگاه در بررسی درستی کار می‌گیرد. عموماً در نهایت نیز مستندات تحویلی یک‌دست نبوده و کاملاً مطابق دستورالعمل داده شده نیستند؛ به‌این دلیل که میکروسافت ورد یک نرم‌افزار حروف‌چین نیست، بلکه یک ویرایشگر پیشرفته است.

اگر دانشجویان از یک ابزار حروف‌چینی همانند لاتک 
\lr{(\LaTeX)}
استفاده کنند، به شرطی که قالب آماده‌ای داشته باشند، لازم نیست نگران دستورالعمل داده شده باشند.
این نوشتار به بیان چنین قالب آماده‌ای برای \پ های دانشگاه حکیم سبزواری می‌پردازد که به همین منظور آماده شده است
\footnote{این قالب با حمایت معاونت پژوهشی دانشگاه حکیم سبزواری آماده شده است.}
.
در صورت استفاده از این قالب، دانشجویان هیچ کاری به دستورالعمل دانشگاه ندارند، تمامی موارد -- همچون اندازه و نوع  قلم متن و عناوین، اندازه حاشیه‌ها، صفحات آغازین و ... -- توسط کلاس آماده شده به صورت خودکار اعمال می‌گردد. دانشجویان و اساتید فقط کافیست روی محتوای کار خود تمرکز نمایند و به چگونگی حروف‌چینی هیچ کاری نخواهند داشت.
شاید دانشجویان در بدو امر مشکلاتی با یادگیری دستورات لاتک داشته باشند، اما به تدریج با یادگیری دستورات اصلی لاتک و مطالعه همین نوشتار و ملاحظه سورس آن، مشکلاتشان برطرف شده و ادامه کار برای آنها بسیار دلنشین و راحت خواهد شد.
